\newpage
\begin{center}
  \textbf{\large Заключение}
\end{center}
\refstepcounter{chapter}
\addcontentsline{toc}{chapter}{Заключение}


В ходе выполнения выпускной квалификационной работы были выполнены поставленные задачи: проведен глубокий анализ существующих подходов, рассмотрены их преимущества и недостатки.

В рамках теоретической части работы был проведен сравнительный анализ подхов по улулчшению качества распознавания речи на трёх русскоязычных наборах аудио данных.
Были изучены особенности использования тех или иных решений.

В прикладной части работы были реализованы обучение пост-коррекции и описание единого пайплайна, проверены возможности использовать одну и ту же LM с разнмыи ASR.
Дополнительно были проверены устойчивость системы к распознаванию аудио плохого качества и проведена оценка используемых вычислительных ресурсов.
Эксперименты показали общее превосходство алгоритма над аналогами.
Ещё были описаны методы регуляризации и текстовой аугментации, показывающие ограничение существующей системы и задающие дальнейшее направление исследований.

По итогу полученная система коррекции текстовых расшифровок аудио показывает высокое качество работы на различных данных и является хорошей основой для дальнейших развития и интеграции.

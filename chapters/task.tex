
{
\newpage
\fontsize{12pt}{12pt}\selectfont
\begin{center}
    \noindent Министерство цифрового развития, связи и массовых коммуникаций Российской Федерации\\Ордена Трудового Красного Знамени \\ 
федеральное государственное бюджетное образовательное учреждение\\
высшего образования 
«Московский технический университет связи и информатики»
\end{center}
\noindent Кафедра \underline{Математическая кибернетика и информационные технологии} \hfill
\vspace{0.1cm}
\noindent
\begin{minipage}[t]{6cm}
«Утверждаю»\\
Зав. кафедрой  МКиИТ \\
Городничев М.Г.\\
« \hrulefill » \hrulefill 2025 г.
\end{minipage}
% \thispagestyle{empty}

\vspace{0.5cm}

\begin{center}
\textbf{З А Д А Н И Е}\\
\textbf{на выпускную квалификационную работу}
\end{center}

\vspace{0.3cm}

\noindent Студенту \underline{Юдину Артёму Андреевичу \hfill } гр. \underline{БТ2101}

\vspace{0.2cm}

\noindent Направление (специальность) \underline{09.03.01 Информатика и вычислительная техника}

\vspace{0.2cm}

\noindent Форма выполнения выпускной квалификационной работы \underline{Бакалаврская работа} \\
Тема выпускной квалификационной работы: \underline{Улучшение качества распознавания речи для рус-}\\\underline{ского языка с использованием языковых моделей}


\noindent Утверждена приказом ректора № \underline{55-с} \hfill от \underline{21.01.2025}

\vspace{0.2cm}

\begin{longtable}{p{0.65\textwidth}|p{0.3\textwidth}}
1. Исходные данные\\
-- Высокоуровневый язык программирования python \\
-- Фреймворк для глубокого обучения PyTorch\\
-- Платформа и библиотеки для работы с моделями и датасетами huggingface \\
\\
2. Содержание расчетно-пояснительной записки (перечень подлежащих разработке вопросов)
& Объем работы в \% и сроки выполнения по разделам\\
Введение & 5\% \hfill 05.04.2025\\
1 Анализ предметной области & 30\% \hfill 16.04.2025\\
Общие сведения об области распознавания речи\\
Методы машинного обучения в аудио\\
Сравнительный анализ методов улучшения качества распознавания речи \\
2 Экспериментальные исследования & 30\% \hfill 02.05.2025 \\
Выбор данных \\
Реализация алгоритмов улучшения качества транскрипций\\
Метрики существующих решений\\
3 Оценка реализованной системы пост-коррекции & 30\% \hfill 27.05.2025\\
Метрики оценки компонента пост-коррекции\\
Метрики оценки устойчивости системы к искажениям\\
Анализ производительности системы\\
Заключение & 5\% \hfill 31.05.2025\\
Список литературы & \\
Приложение& \\
\end{longtable}

\noindent 7. Консультанты по ВКР (с указанием относящихся к ним разделов проекта):
\vspace{2em}

\noindent
\begin{tabular}{@{}p{0.4\textwidth}p{0.1\textwidth}p{0.45\textwidth}@{}}
 &  & \hrulefill \hspace{2em}\hrulefill\\
 & & \centering\footnotesize (подпись)\hspace{8em}(ФИО)\\
 
\end{tabular}

\vspace{1.5em}

\noindent 8. Срок сдачи студентом законченной ВКР: \hrulefill\\
 \hspace{1em}Дата выдачи задания: \hrulefill\\
 \hspace{1em}Руководитель\underline{\hspace{7cm}} \hspace{1em}Городничев Михаил Геннадьевич\\
 \hspace*{5cm}(\textit{подпись}) \hspace{5cm}(\textit{Ф.И.О.})\\
 \noindent \underline{\hspace{7cm}}\underline{штатная}\underline{\hspace{6cm}}нагрузка\\
\hspace*{5cm}(\textit{штатная или почасовая})\\

\noindent Задание принял к исполнению\underline{\hspace{11cm}}\\
\hspace*{9cm}(\textit{подпись студента})\\
\begin{center}
    Примечание: Настоящее задание прилагается к законченной ВКР
\end{center}
}
\thispagestyle{empty}
\clearpage
\setcounter{page}{3}
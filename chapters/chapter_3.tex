\newpage
\begin{center}
  \textbf{\large 3. РАЗРАБОТКА СОБСТВЕННОГО МЕТОДА}
\end{center}
\refstepcounter{chapter}
\addcontentsline{toc}{chapter}{3. РАЗРАБОТКА СОБСТВЕННОГО МЕТОДА}

Наша основная идея в обучение языковой модели на русском языке для исправление ошибок.
Как мы обсуждали ранее, в основном методы, завязанные на LM полагаются на скрытое представление ASR модели или же на её проекцию в скрытое представление языковой модели.
Мы же будем проводить наши тесты с использованием естественного языка.
То есть мы будем сначала транскрибировать аудио в текст и только потом его править.
Таким образом мы решим проблему универсальности подхода и независимости ASR-LM пары.

В качестве двух моделей мы возьмём ruT5 и Fred-T5.
Обе модели были разработаны Сбером и обучены на корпусе русского языка.
Они построены на архитектуре энкодер-декодер трансформеров.
Их мы выбрали по нескольким причинам:

\begin{enumerate}
  \item \textbf{Размер}.
  Мы берём Base (220М) и distil (95М) версии соответственно. 
  В паре с fastconformer это займёт немногим больше 3 Гб видеопамяти, что позволит влезть в любую пользовательскую видеокарту.
  \item \textbf{Архитектура}.
  Энкодер-декодер лучше улавливает ошибки и корректирует их.
  Такие выводы были сделаны на основе решения этой задачи для английского языка, где в качестве моделей-корректоров использовались Mistral-7b и Llama3-8b.
\end{enumerate}

Теперь необходимо создать датасет для обучения.
Для этого возьмём и сгенерируем 5 гипотез с помощью BeamSearch метода.
Затем мы подадим все 5 гипотез в нашу модель и попросим сгенерировать исправленную последовательность.
Обучение ruT5 будет проходить в few-shot сценарии, когда мы ей будем показывать 3 примера из валидационной выборке в качестве примера для улучшения.
Так мы проведём 15 обучающих эпох и сохраним модель с наименьшим WER на валидационной выборке.

Почему подавать 5 гипотез в модель? Что мы можем взять из гипотез?
Почему просто не реранжировать гипотезы?
Давайте проанализируем полученные данные.

\textbf{Тут графки.}

Сначала разъясним, почему не реранжирование.
Первый график на Рис. 1 показывает процент случаев, когда гипотеза 2-5 имеют ниже WER, чем 1 гипотеза.
Их процент колеблется от 4 до 8, что значит мы можем не получить особого прироста.
Второй график же показывает процент токенов, которые:
\begin{itemize}
  \item Есть в праильной транскрипции.
  \item Есть в гипотезе $n_k$ и нет в гипотезах $n_i$, где $i < k$.
\end{itemize}

Это та самая информация, которую наша модель может извлечь из сгенерированных гипотез и улучшить качество транскрипции.
Именно поэтом мы переходим к обучению отдельной языковой модели на ошибках.

Обучение будем проводить с использованием метода Low-rank adaptaion (LoRA).
Это метод, при котором изначальные веса модели замораживаются и никак не изменяются.
Далее мы выбираем слои модели, которые мы будем изменять. В нашем случае это были слои attention.
Затем у каждой матрицы слоя attention (Q, K, V, O) инициализируется своя пара матриц A и B размерами (h, rank) и (rank, h), где h – размер скрытого пространства матрицы attention, а rank – размер ранга, заданный при LoRA.
И так во время обучения изменяются именно матрицы A и B, которые как бы представляют собой сжатые слои. Чем больше ранг матрицы, тем более точное подобие attention мы получим, но вместе с тем больше памяти потребуется для обучения. Размер batch'а был задан равным 64, использовался линейный планировщик с разогревом.
Также стоит сказать, что модель обучалась исключительно на ошибках Fastconformer модели.
Соответственно результаты для модели Whisper в связке с ruT5 будут скорее результатов переноса способностей к коррекции между разными моделями в рамках одного набора данных.

Помимо обученного T5 мы применяли Fred-T5 дистилированный, который используется как инструмент для восстановления знаком препинания.
Это сделано для проверки способностей ASR моделей генерировать последовательности правильные с точки зрения пунктуации.

% Please add the following required packages to your document preamble:
% \usepackage{multirow}
% Please add the following required packages to your document preamble:
% \usepackage{multirow}
\begin{table}[]
\centering
\caption{Comparison of WER on a training CV21 and development sets.}
\begin{tabular}{|c|c|ccc|}
\hline
\multirow{2}{*}{Модель}        & \multirow{2}{*}{Сеттинг}             & \multicolumn{3}{c|}{Датасет, WER}                                \\ \cline{3-5} 
                               &                                      & \multicolumn{1}{c|}{CV21}  & \multicolumn{1}{c|}{RuLS} & OpenSTT \\ \hline
\multirow{6}{*}{Fastconformer} & Greedy                               & \multicolumn{1}{c|}{12.31} & \multicolumn{1}{c|}{-}    & -       \\ \cline{2-5} 
                               & BeamSearch                           & \multicolumn{1}{c|}{12.26} & \multicolumn{1}{c|}{-}    & -       \\ \cline{2-5} 
                               & LM                                   & \multicolumn{1}{c|}{9.87}  & \multicolumn{1}{c|}{-}    & -       \\ \cline{2-5} 
                               & Greedy+Distill                       & \multicolumn{1}{c|}{13.81} & \multicolumn{1}{c|}{-}    & -       \\ \cline{2-5} 
                               & BeamSearch+Distill                   & \multicolumn{1}{c|}{3.82}  & \multicolumn{1}{c|}{-}    & -       \\ \cline{2-5} 
                               & BeamSearch+ruT5                      & \multicolumn{1}{c|}{3.20}  & \multicolumn{1}{c|}{-}    & -       \\ \hline
\multirow{4}{*}{Whisper}       & Greedy                               & \multicolumn{1}{c|}{19.60} & \multicolumn{1}{c|}{-}    & -       \\ \cline{2-5} 
                               & BeamSearch                           & \multicolumn{1}{c|}{18.33} & \multicolumn{1}{c|}{-}    & -       \\ \cline{2-5} 
                               & Greedy+Distill                       & \multicolumn{1}{c|}{20.57} & \multicolumn{1}{c|}{-}    & -       \\ \cline{2-5} 
                               & BeamSearch+ruT5                      & \multicolumn{1}{c|}{4.99}  & \multicolumn{1}{c|}{-}    & -       \\ \hline
\end{tabular}
\label{tab:res_cv_trained}
\end{table}

Взглянув на Таблицу~\ref{tab:res_cv_trained}, мы можем сделать несколько выводов.
Во-первых — подход с коррекцией текстовой расшифровки аудио показывает себя лучше всех.
Это значит, что богатые лингвистические знания модели + дообучение под нашу задачу работают отлично.
Во-вторых можно увидеть, что перенос знаний между разными ASR моделями также работает.
Так наша модель смогла в 4 раза улучшить качество распознания для Whisper, хотя до этого она обучалась исключительно на результатах Fastconformer.
Также в обоих случаях использование дистиллированной модели при сценарии жадного декодирования не принесло никакого прироста в качестве, а наоборот показало ухудшение. Причин такого падения назвать наверняка нельзя.
Можно сказать, что ASR модели, используя свои знания о языке, имеют более точное представление о правописание и пунктуации.
Или же distill модель пыталась поменять какое-то слово, которое в изменение не нуждалось, из-за чего WER вырос.

По итогу можно уверенно сказать, что наш метод показал в разы лучшее качество в сравнение с другими методами и при этом затраты вычислительных ресурсов, требуемых для inference пары Fastconformer + ruT5 сравнительно не велико.
В сумме обе эти модели помещаются в 3-4 Гб видеопамяти, что в современном мире является минимумом для самых низкобюджетных карт, в том числе и среди ноутбучных.

% В данной главе используется моделирование молекулярной динамики для расчета фазовых диаграмм обобщенных систем Леннарда-Джонса с различными показателями притяжения.
% Оценены коэффициенты диффузии и подвижности (обратной диффузии) и проанализированы спектры коллективного возбуждения на жидких бинодалиях. 
% Отмечено, что зависимость коэффициента подвижности от температуры является линейной в широком диапазоне температур, а ее наклон увеличивается с увеличением показателя притяжения.
% При отклонении подвижности от линейной зависимости дисперсионные соотношения коллективных возбуждений жидкости переходят от осцилирующего к монотонному виду.

% \section{Роль диффузии в науке и технике}
% \label{MACR-SecIntroduction}

% Диффузия играет решающую роль в различных процессах переноса массы, начиная с науки и техники и заканчивая живой природой.
% Диффузия выступает ключевым механизмом в биологических процессах~\cite{10.1016/j.bbagen.2013.09.037, 10.1038/s41598-018-22643-9}, а также в кинетике химических реакций.
% Знание механизмов диффузии позволит добиться значительного прогресса в новых биотехнологиях и медицине, решить важные проблемы химической и фармакологической промышленности~\cite{10.1002/3527602836}.

% Процесс диффузии очень хорошо изучен в газах и твердых телах.
% Например, в кристаллических системах~\cite{10.1016/0079-6816(95)00039-2}, что связано с ее практической ценностью в металлургии для легирования~\cite{10.1016/s0924-0136(96)02826-9, 10.1016/j.actamat.2015.10.010, 10.1134/s1063783411110308} и эксплуатации полупроводниковой электроники~\cite{10.1103/physrevlett.84.4220, 10.1016/j.physrep.2009.10.003}.

% В данной главе, используя метод молекулярной динамики (МД), моделируются обобщенные системы Леннарда-Джонса с различными показателями притяжения.
% Рассчитаны температурные зависимости подвижности частиц (коэффициента обратной диффузии) на бинодали жидкость-газ. 
% Здесь же рассмотрена связь между диффузией, дальнодействием межчастичного притяжения и свойствами коллективных возбуждений в простых жидкостях.

% \section{Методы}
% \label{MACR-SecMethods}

% \subsection{Расчет фазовых диаграмм методами молекулярной динамики}
% \label{MACR-SubSecMD}

% В этой главе анализируются транспортные свойства и их связь с коллективными модами на жидких бинодальях.
% Для систем, взаимодействующих через обобщенный потенциал Леннарда-Джонса (LJ$n$-$m$):
% \begin{equation}
%   U_{n-m}(r)=4 \varepsilon\left[\left(\frac{\sigma}{r}\right)^{n}-\left(\frac{\sigma}{r}\right)^{m}\right]
%   \label{MACR-eq1}
% \end{equation}
% где $\epsilon$ и $\sigma$ -- характерные масштабы энергии и длины соответственно.
% На протяжении всей статьи используются приведенные единицы измерения температуры $ T/ \epsilon \rightarrow T $, расстояния $ r/ \sigma \rightarrow r $ и плотности $ \rho \sigma ^ 3 \rightarrow n$.


% Были рассмотрены потенциалы LJ$12$-$4$, LJ$12$-$5$, LJ$12$-$6$ и LJ$16$-$6$.
% Чтобы сравнить полученные результаты для LJ$n$-$m$ с результатами для системы, в которой взаимодействия не являются сферически-симметричными, были также смоделировали этан~\cite{10.1021/acs.jced.6b01036}.
% В выбранной модели молекула этана рассматривается как пара жестко связанных радикалов CH$_3$, взаимодействующих с радикалами других молекул через потенциал~\cite{10.1021/acs.jced.6b01036}:
% \begin{equation}
%   U_{\rm {ethane}}(r) = \tilde \varepsilon\left[\left(\frac{\sigma}{r}\right)^{16}-\left(\frac{\sigma}{r}\right)^{6}\right],
%   \label{MACR-eq2}
% \end{equation}
% где $\tilde\varepsilon = 0,69396$ ккал/моль и $\sigma = 3,783$\AA.

% Все МД-симуляции были выполнены в ансамбле NVT (N, V и T --количество частиц, объем системы и температура соответственно) с периодическими граничными условиями с использованием пакета моделирования \\  LAMMPS~\cite{10.1006/jcph.1995.1039}.
% В первую очередь были рассчитаны линии бинодали~\cite{10.1021/jp806127j, 10.1021/jp1117213}.
% Исходное состояние системы формировалось в два этапа: (i) кубическая область моделирования заполнялась равновесным кристаллом (в нашем случае ГЦК) из $N$ частиц с плотностью, соответствующей близкому к нулю давлению; 
% (ii) область моделирования была расширена в направлении осей $x$ так, чтобы окончательная средняя плотность системы $\rho_a$ стала равной значениям, указанным в таблице~\ref{MACR-Table1}.
% Результирующее начальное состояние показано на рис.~\ref{MACR-Figure1}(а). 
% Затем температура системы линейно увеличивалась от $T_{start}$ до $T_{stop}$ в течение $n_{step}$ шагов моделирования с временным шагом $\Delta t$.
% Конденсированная фаза в какой-то момент начинает испаряться, образуя сосуществование газа и конденсата, если температура ниже критической, как показано на рис.~\ref{MACR-Figure1}(b).
% Принципиально то, что полученное таким образом состояние системы почти всегда имеет границы фаз, ортогональные оси $x$.
% В результате плотности $\rho_g$ и $\rho_c$ газовой и конденсированной фаз соответственно могут быть рассчитаны путем подгонки профиля плотности $\rho(x)$ выражением~\cite{10.1021/jp806127j, 10.1021/jp1117213}:

% \begin{equation}
%   \rho(x)=\frac{\rho_{l}+\rho_{g}}{2}-\frac{\rho_{l}-\rho_{g}}{2} \tanh \left(\frac{|x|-L}{\delta}\right),
%   \label{MACR-eq3}
% \end{equation}
% где $L$ — половина длины области моделирования, занимаемой жидкой фазой, а $\delta$ — характерная ширина границы раздела.
% Пример профиля плотности системы и его аппроксимация уравнением~\eqref{MACR-eq3} показаны на рис.~\ref{MACR-Figure1}(c) гистограммой и красной линией соответственно.
% Параметры моделирования для рассмотренных моделей сведены в табл.~\ref{MACR-Table1}.

% \begin{table}[]
%   \centering
%   \begin{tabular}{|lllllcl|}
%     \hline
%     \multicolumn{1}{|l|}{Potential} & \multicolumn{1}{l|}{$\rho_a$} & \multicolumn{1}{l|}{$r_c$} & \multicolumn{1}{l|}{$T_{start}$} & \multicolumn{1}{l|}{$T_{stop}$} & \multicolumn{1}{l|}{$n_{step}$}                       & $\Delta t$                          \\ \hline
%     \multicolumn{7}{|c|}{Значения в безразмерных единицах:}                                                                                                                                                                                                         \\ \hline
%     \multicolumn{1}{|l|}{LJ12-4}    & \multicolumn{1}{l|}{0.25}     & \multicolumn{1}{l|}{15.0}  & \multicolumn{1}{l|}{1.0}         & \multicolumn{1}{l|}{5.5}        & \multicolumn{1}{c|}{\multirow{4}{*}{$3 \times 10^6$}} & \multirow{4}{*}{$5 \times 10^{-4}$} \\ \cline{1-5}
%     \multicolumn{1}{|l|}{LJ12-5}    & \multicolumn{1}{l|}{0.25}     & \multicolumn{1}{l|}{10.0}  & \multicolumn{1}{l|}{0.8}         & \multicolumn{1}{l|}{2.4}        & \multicolumn{1}{c|}{}                                 &                                     \\ \cline{1-5}
%     \multicolumn{1}{|l|}{LJ12-6}    & \multicolumn{1}{l|}{0.35}     & \multicolumn{1}{l|}{8.0}   & \multicolumn{1}{l|}{0.5}         & \multicolumn{1}{l|}{1.4}        & \multicolumn{1}{c|}{}                                 &                                     \\ \cline{1-5}
%     \multicolumn{1}{|l|}{LJ16-6}    & \multicolumn{1}{l|}{0.31}     & \multicolumn{1}{l|}{8.0}   & \multicolumn{1}{l|}{0.8}         & \multicolumn{1}{l|}{1.6}        & \multicolumn{1}{c|}{}                                 &                                     \\ \hline
%     \multicolumn{7}{|c|}{Единицы измерения СИ:}                                                                                                                                                                                                                     \\ \hline
%     \multicolumn{1}{|l|}{Ethane}    & \multicolumn{1}{l|}{$0.22\mathrm{\frac{g}{cm^3}}$}     & \multicolumn{1}{l|}{$25\text{\AA}$}    & \multicolumn{1}{l|}{$80\,\mathrm{K}$}          & \multicolumn{1}{l|}{$320\,\mathrm{K}$}        & \multicolumn{1}{l|}{$2 \times 10^6$}                  & $2\,\mathrm{\text{фс}}$                                   \\ \hline
%   \end{tabular}
%   \caption{Параметры, используемые в МД-моделировании для бимодальных расчетов: где $\rho$ — средняя плотность системы, $r_c$ — радиус отсечки, $T_{start}$ и $T_{stop}$ — начальная и конечная температуры моделирования, соответственно, $n_{step}$ — количество шагов моделирования, а $\Delta t$ — временной шаг.}
%   \label{MACR-Table1}
% \end{table}


% \begin{figure}[!t]
%   \centering
%   \includegraphics[width=150mm]{MACR-Figure1.png}
%   \caption{(a) Система частиц для расчета фазовой диаграммы.
%     Система частиц с потенциалом взаимодействия LJ12-6 при температуре $T=1.13$ в виде плоского слоя.
%     (b) Профиль плотности системы вдоль оси $x$.
%     Область с высокой плотностью представляет собой конденсат, с низкой -- газ.
%     Темно-красная линия представляет собой аппроксимацию профиля плотности уравнением~\eqref{MACR-eq3}.}
%   \label{MACR-Figure1}
% \end{figure}


% \begin{figure}[!t]
%   \centering
%   \includegraphics[width=150mm]{MACR-Figure2.pdf}
%   \caption{Фазовая диаграмма системы LJ12-6.
%     Оранжевым и синим цветами обозначены символы плотности газа и конденсата соответственно, полученные путем подгонки данных МД по уравнению~\eqref{MACR-eq3}.
%     Зеленые символы -- медиана $\rho_m=(\rho_g+\rho_c)/2$.
%     Сплошная красная линия соответствует уравнению~\eqref{MACR-eq4}.
%     Тройные и критические точки -- синие и красные звездочки соответственно.
%   }
%   \label{MACR-Figure2}
% \end{figure}

% Вблизи критической температуры расчет плотности газа и жидкости становится затруднительным из-за усиленных флуктуаций плотности.
% Однако положение критической точки на фазовой диаграмме можно вычислить, аппроксимируя жидкостную и газообразную бинодальные ветви вблизи критической точки выражением~\ref{MACR-eq4}.

% В трехмерии критический индекс $\beta_c = 0,5$ для потенциала LJ$12-4$, тогда как $\beta_c = 0,325$ для LJ$12$-$5$, LJ$12$-$6$, LJ$16$-$6$ и этана, согласно предыдущим результатам~\cite{10.1021/acs.jced.6b01036,10.1021/jp9072137,10.1103/physrevlett.89.025703}.

% Пример полученных бинодалей для LJ12-6 и их аппроксимации уравнением~\eqref{MACR-eq4} показаны на рис.~\ref{MACR-Figure2}.
% Обратите внимание, что на конденсированной бинодали имеется явный излом (см. рис.~\ref{MACR-Figure2}), который указывает на падение плотности при плавлении и соответствует положению тройной точки.
% Полученные значения $A$ и $a$ из уравнения~\eqref{MACR-eq4}, а также значения плотности и температуры критических и тройных точек для рассматриваемых систем представлены в таблице~\ref{MACR-Table2}.

% \begin{figure}[!t]
%   \centering
%   \includegraphics[width=\linewidth]{MACR-Figure3.pdf}
%   \caption{(а) Фазовые диаграммы рассматриваемых систем. 
%     Фазовые диаграммы рассчитывались методом двухфазного моделирования, который описан в разделе~\ref{MACR-SecMethods}.
%     Цветные точки обозначают рассчитанные бинодали, треугольники -- срединные точки.
%     Сплошные серые кривые показывают диапазон температур, используемый для аппроксимации и определения параметров в уравнении ~\eqref{MACR-eq4}.
%     Штриховые серые кривые соответствуют экстраполированным биноидам.
%     (б) Температурная зависимость подвижности частиц.
%     Подвижность частиц была рассчитана на жидких бинодальях с использованием метода, описанного в разделе~\ref{MACR-SecMethods}.
%     Точки, соответствующие экстраполированным бинодалим, отмечены серым цветом. 
%     Прямые линии -- линейная аппроксимация подвижности.
%     На вставке показана расчетная подвижность метана.}
%   \label{MACR-Figure3}
% \end{figure}

% \subsection{Расчет диффузии и спектров на бинодали}

% Далее для расчета подвижности на конденсированной бинодали моделировались системы с плотностью и температурой, взятыми из полученных фазовых диаграмм.
% Обобщенные систем Леннарда-Джонса с $N = 4,0 \times 10 ^ 3$ моделировались с шагом по времени $1,5 \times 10 ^ 5$.
% Для этана использовались $N = 1,065 \times 10 ^ 4 $ молекул и проведены моделирования с шагом по времени $7.0 \times 10^5 $.
% Для релаксации системы использовались первые $ 5.0 \times 10 ^ 4 $ временных шагов для обобщенных LJ-систем и $ 5.0 \times 10 ^ 5 $ для этана.
% Остальные параметры были такими же, как и при расчете фазовых диаграмм.

% Коэффициент самодиффузии $D$ определялся по среднеквадратичному отклонению частиц:
% \begin{equation}
%   \sigma^2(t) = \sum\limits_{\alpha = 1}^{N} (r_{\alpha}(t) - r_{\alpha}(0))^2 / N, \quad \sigma^2(t) = 6Dt,
%   \label{MACR-eq5}
% \end{equation}
% где $\sigma$ — среднеквадратичное отклонение, а $t$ — время.
% Подвижность $\mu$ связана с коэффициентом диффузии соотношением Эйнштейна
% \begin{equation}
%   \mu = \frac{D}{T},
%   \label{MACR-eq6}
% \end{equation}
% где $T$ — температура системы.

% Наконец, спектры возбуждения были получены с использованием обработки тока скорости~\cite{10.1063/1.5050708}:
% \begin{equation}
%   C_{L, T}(\mathbf{q}, \omega)=\int dt e^{i \omega t} \text{Re} \left\langle\mathbf{j}_{L, T}(\mathbf{q}, t) \mathbf{j}_{L, T}(-\mathbf{q}, 0)\right\rangle,
%   \label{MACR-eq7}
% \end{equation}
% где ${\bf k}$ и $\omega$ — волновой вектор и частота,
% $\mathbf{j}_{L}=\mathbf{q}(\mathbf{j} \cdot \mathbf{q} ) / q^{2}$ и $\mathbf{j}_{T}=(\mathbf{j \cdot e_{\perp})e_{\perp}}$ — продольная ($L$) и поперечная ($T$) компоненты тока частиц,\\
% $\mathbf{j}(\mathbf{q}, t)=N^{-1} \sum_{s} \mathbf{v}_{s}(t) \ exp \left(i \mathbf{q} \mathbf{r}_{s}(t)\right)$ и $\mathbf{v}_{s}(t)=\dot{\mathbf{r} }_{s}(t)$ — скорость $s$-й частицы.
% Суммирование ведется по всем $N$ частицам в системе.
% Усреднение по каноническому ансамблю обозначается $\langle\cdots\rangle$. 
% Анализ $C_{L, T}(\mathbf{q}, \omega)$ проводился с помощью методов из работы~\cite{10.1038/s41598-019-46979-y}, что позволило получить дисперсионные соотношения продольной и поперечной мод.

% МД-моделирование для расчета спектров возбуждения отличается от моделирования для подвижности только длительностью временного шага. Для LJ$12$-$4$ и LJ$16$-$6$ шаг по времени был выбран как $\Delta t = 1 \times 10 ^ {-4} \sqrt {m \sigma ^ 2 / \epsilon}$, а для LJ$12$-$5$ и LJ$12$-$6$ -- $\Delta t = 5 \times 10 ^ {-4} \sqrt {m \sigma ^ 2 / \epsilon}$.

% \begin{figure}[!t]
%   \centering
%   \includegraphics[width=160mm]{MACR-Figure4.pdf}
%   \caption{(a) Температурная зависимость подвижности системы LJ$12$-$6$ вдоль жидкостной бинодали.
%     Температуры, при которых рассчитывались спектры возбуждения, указаны черными стрелками.
%     (b) - (f) спектры возбуждения LJ$12$-$6$ систем.
%     Спектры рассчитывались путем анализа скорости течения (уравнение~\eqref{MACR-eq7}) так же, как в Ref.~\cite{10.1038/s41598-019-46979-y}.
%     Красный цвет соответствует гибридным модам, серый — результатам анализа отдельных мод~\cite{10.1038/s41598-019-46979-y}.
%     В левом верхнем углу указаны пониженные температуры.}
%   \label{MACR-Figure4}
% \end{figure}

% \section{Результаты}
% \label{MACR-SecResults}

% Результаты расчета границ сосуществования газа и жидкости показаны на рис.~\ref{MACR-Figure3}(а).
% Цветные точки обозначают бинодали, треугольники -- срединные точки.
% Точки, которые использовались для аппроксимации [с использованием уравнения~(\ref{MACR-eq4})], выделены сплошной серой линией.
% Экстраполированные бинодали обозначены пунктирной серой линией.
% Для каждой рассматриваемой системы температура и плотность выражаются в единицах температуры и плотности тройной точки соответственно.
% Последние значения вместе с параметрами критических точек приведены в Таб.~\ref{MACR-Table2}.

% \begin{table}[h!]
%   \centering{
%     \begin{tabular}{C{1.5cm}|C{1.0cm}|C{1.0cm}|C{1.0cm}|C{1.0cm}|C{1.0cm}|C{1.0cm}}
%       LJn-m & $T_{\rm CP}$ & $\rho_{\rm CP}$ & $T_{\rm TP}$ & $\rho_{\rm TP}$ & $A$ & $a$ \\ \hline
%       LJ12-4 & 4.85 & 0.291 & 1.75 & 0.952 & 0.559 & 0.107 \\
%       LJ12-5 & 2.18 & 0.304 & 1.03 & 0.867 & 0.804 & 0.208 \\
%       LJ12-6 & 1.29 & 0.315 & 0.72 & 0.830 & 1.002 & 0.326 \\
%       LJ16-6 & 1.55 & 0.316 & 0.98 & 0.816 & 0.969 & 0.334 \\
%       Ethane & 305.3 & 206.7 & 90.34 & 651.9 & 113.1 & 1.158
%     \end{tabular}

%   }
%   \caption{Значения плотностей и температур критических и тройных точек и параметры аппроксимации по уравнению~\eqref{MACR-eq4} для рассматриваемых моделей.
%     Для обобщенных систем LJ температуры и плотности даны в сокращенных единицах.
%     Для этана температура выражена в К, а плотность выражена в $\text{кг}/\text{м}^3$.
%     Параметры критической и тройной точек для этана взяты из работы~\cite{10.1063/1.555785}.}
%   \label{MACR-Table2}
% \end{table}

% \begin{figure}[!t]
%   \centering
%   \includegraphics[width=160mm]{MACR-Figure5.pdf}
%   \caption{Результаты для потенциала LJ$12$-$4$.
%     Рисунок аналогичен рисунку~\ref{MACR-Figure4}(a)-(f).}
%   \label{MACR-Figure5}
% \end{figure}


% Замечено, что с увеличением дальнодействия потенциала увеличиваются как температуры тройных и критических точек, так и их отношение $T_{\rm CP}/T_{\rm TP}$.

% Затем по рассчитанным фазовым диаграммам была вычислена подвижность частиц при плотностях и температурах, соответствующих бинодали жидкости.
% Полученная зависимость подвижности частиц от температуры представлена на рис.~\ref{MACR-Figure3}(b).
% Цветные точки на (b) соответствуют цветным точкам на (a).
% Серые точки обозначают подвижности на экстраполированных частях бинодали.

% \begin{figure}[!t]
%   \centering
%   \includegraphics[width=160mm]{MACR-Figure6.pdf}
%   \caption{Результаты для потенциала LJ$12$-$5$.
%     Рисунок аналогичен рисунку~\ref{MACR-Figure4}(a)-(f).}
%   \label{MACR-Figure6}
% \end{figure}

% Отметим, что при низких температурах подвижность на бинодали имеет линейную зависимость от температуры.
% Ее наклон увеличивается с уменьшением дальнодействующего характера потенциала взаимодействия (т.е. с увеличением показателя притяжения).
% Линейная зависимость сохраняется до определенной температуры, после которой происходит отклонение от линейной зависимости.
% Возникновение такой нелинейности может быть связано с особенностями коллективной динамики частиц, которые должны коррелировать со спектрами коллективных возбуждений.

% Вычисленные спектры системы LJ$12$-$6$ показаны на рис.~\ref{MACR-Figure4}.
% На рисунке~\ref{MACR-Figure4}(а) изображены зависимости подвижности от температуры, а черными стрелками указаны температуры, при которых рассчитывались спектры. 
% Были выбраны точки вблизи температуры, при которой наблюдается начало нелинейной зависимости, а также температура вблизи тройной точки.
% На рисунке~\ref{MACR-Figure4}(b)-(f) показаны вычисленные дисперсии продольных и поперечных мод в этих точках.
% Красный цвет соответствует модели с двумя осцилляторами, а серый — одномодовому анализу~\cite{10.1038/s41598-019-46979-y}.

% Нетрудно заметить, что по мере приближения температуры к точке, соответствующей возникновению нелинейной зависимости, дисперсионные соотношения демонстрируют переход от осциллирующей к монотонной зависимости от волнового числа.
% Таким образом, качественное изменение температурной зависимости подвижности частиц сопровождается изменением спектров возбуждения.
% Наблюдаемая картина не является особенной для системы LJ$12$-$6$.
% Аналогичная тенденция замечена и в других исследованных обобщенных ЛД-систем.
% Это наблюдение дает новое свидетельство тесной связи между диффузией и свойствами коллективных возбуждений.


% \begin{figure}[!t]
%   \centering
%   \includegraphics[width=160mm]{MACR-Figure7.pdf}
%   \caption{Результаты для потенциала LJ$16$-$6$.
%     Рисунок аналогичен рисунку~\ref{MACR-Figure4}(a)-(f).}
%   \label{MACR-Figure7}
% \end{figure}


% \section{Заключение главы}
% \label{MACR-SecConclusions}

% В данной главе исследовано влияние формы потенциала парного взаимодействия на фазовые диаграммы и подвижность частиц в жидкой фазе.
% Были рассчитаны кривые сосуществования газа и жидкости для потенциалов с переменным дальнодействием притяжения.
% Отмечено, что с увеличением дальнодействующего характера потенциала температуры тройной и критической точек, а также их отношение $T_{\rm CP}/T_{\rm TP}$ увеличиваются.
% Коэффициент диффузии и обратный ему коэффициент подвижности вычислялись на жидких бинодалиях.
% Было обнаружено, что температурная зависимость подвижности линейна в широком диапазоне температур с тем большим наклоном, чем меньше диапазон притяжения.
% Кроме того, установлено, что начало нелинейной температурной зависимости подвижности при высоких температурах совпадает с переходом дисперсионных зависимостей коллективных возбуждений от осциллирующей к монотонной зависимости от волнового числа.
% Полученные результаты дают возможность для дальнейшего изучения диффузии и ее связи с коллективными процессами в конденсированных многочастичных системах.

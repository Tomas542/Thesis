% \documentclass[a4paper, 12pt]{article}
% \usepackage[russian]{babel}
% \usepackage[utf8]{inputenc}
% \usepackage{graphicx}
% \usepackage{booktabs}
% \usepackage{array}
% \usepackage{tabularx}
% \usepackage{setspace}
% \usepackage[T2A]{fontenc}
% \usepackage{indentfirst}
% \usepackage{sectsty}
% \usepackage{diagbox}
% \usepackage{enumitem}
% \usepackage{amsmath}
% \usepackage{amssymb}
% \usepackage{amsfonts}
% \usepackage{fontspec}
% \setmainfont{Times New Roman}
% \newfontfamily\cyrillicfont{Times New Roman}
% \usepackage{setspace}
% \onehalfspacing

% \usepackage{microtype} 

% \allsectionsfont{\bfseries}
% \usepackage[a4paper, left=3cm, right=1cm, top=2cm, bottom=2cm]{geometry}
% \setlength{\parindent}{1.25cm}

\newpage

\thispagestyle{empty}
\begin{center}
Министерство цифрового развития, связи и массовых коммуникаций\\
Российской Федерации\\
Орденa Трудового Красного Знамени федеральное государственное бюджетное образовательное учреждение высшего образования\\
«Московский технический университет связи и информатики»
\end{center}
\vspace{0.5cm}
\noindent Разрешаю\\
\noindent допустить к защите\\
\noindent Зав. кафедрой МКиИТ\\
\noindent \underline{Городничев М.Г.}\\
\noindent \underline{\hspace{3.5cm}} 2025 г.
\vspace{1cm}

\begin{center}
\fontsize{18pt}{20pt}\selectfont
\textbf{ВЫПУСКНАЯ КВАЛИФИКАЦИОННАЯ РАБОТА}\\
\fontsize{12pt}{20pt}\selectfont
НА ТЕМУ\\
\fontsize{16pt}{20pt}\selectfont
Улучшение качества распознавания речи для русского языка с использованием языковых моделей
\end{center}

\vspace{0pt plus3fill}
\fontsize{14pt}{14pt}
\begin{tabular}{@{}ll@{}}
Студент: & \underline{\hspace{3cm}} Юдин Артём Андреевич \\
Руководитель: & \underline{\hspace{3cm}} Городничев Михаил Геннадьевич
\end{tabular}

\vspace{0pt plus6fill}
\begin{center}
    Москва, 2025
\end{center}
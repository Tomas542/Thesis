\newpage
\clearpage
\begin{center}
    \textbf{ОТЗЫВ РУКОВОДИТЕЛЯ}
\end{center}

на выпускную квалификационную работу студента Кравченко Николая Алексеевича на тему Улучшение качества распознавания речи для русского языка с использованием языковых моделей.

Выпускная квалификационная работа Кравченко Н.А. посвящена актуальной теме разработки системы обработки естественного языка на основе больших языковых моделей с использованием подхода генерации, дополненной поиском.

Выпускная квалификационная работа состоит из трех глав, введения, заключения и списка использованной литературы. Во введении студентом представлено описание предметной области, обоснована актуальность темы, а также сформулированы цель и задачи исследования. Оформление ВКР соответствует установленным требованиям.

В первой главе Кравченко Н.А. провёл детальный анализ предметной области, рассмотрел историю развития технологий вопросно-ответных систем, а также ключевые подходы к реализации RAG. Во второй главе описан процесс разработки и тестирования RAG-систем с использованием различных языковых моделей, проведена оценка их качества на наборе данных XQuAD.ru. В третьей главе Кравченко Н.А. представил результаты оценки качества разработанных RAG-систем.

Работа выполнена с использованием современного технологического стека. Практическая значимость исследования заключается в возможности применения разработанных подходов для улучшения качества ответов в русскоязычных системах на основе больших языковых моделях.

Кравченко Н.А. продемонстрировал высокий уровень ответственности, самостоятельности и технической подготовки. Им были реализованы алгоритмы, которые в полной мере соответствуют заявленным целям и задачам исследования. Замечаний к работе нет.

\newpage
Учитывая изложенное, считаю, что выпускная квалификационная работа выполнена на высоком уровне, соответствует предъявляемым требованиям, может быть допущена к защите, а её автор заслуживает присвоения квалификации бакалавра по направлению 09.03.01 «Информатика и вычислительная техника».
\thispagestyle{empty}

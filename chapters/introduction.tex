\newpage
\begin{center}
  \textbf{\large АННОТАЦИЯ}
\end{center}


Объектом исследования данной работы стали большие языковые модели, а также модели распознания речи и их взаимодействие.
С помощью данных моделей исследовались методы повышения точности транскрипций, перенос знаний о коррекции ошибок между разными моделями распознания речи и наборами данных.
Для решения поставленной задачи методом пост-коррекции и рескоринга были обучены языковые модели.
Все эксперименты проводились с использованием Python и библиотек Transformers, NeMO и Hydra.

В ходе работы были обучены и протестированы различные методы, выявлены их сильные и слабые стороны.
Посчитаны результаты на различных наборах данных.
В качестве основной метрики оценки использовалась Word Error Rate.
Предложен новый метод улучшения качества, способный работать с разными аудио моделями.
На основании полученных результатов, были сделаны вывод о переносе знаний об ошибках между разными моделями, пригодности языковых моделей для такого рода задач, влияние на скорость, точность и ресурсоёмкость задачи, а также предложены методы для дальнейшего улучшения метода.
Был разработан алгоритм обучения языковых моделей на основе ошибок систем распознания речи и дальнейшего использования этих моделей для пост-коррекции.

\onehalfspacing
\setcounter{page}{2}

\newpage
\renewcommand{\contentsname}{\centerline{\large СОДЕРЖАНИЕ}}
\tableofcontents

\newpage
\begin{center}
  \textbf{\large ВВЕДЕНИЕ}
\end{center}
\addcontentsline{toc}{chapter}{ВВЕДЕНИЕ}


\textbf{Актуальность}

За прошедшие годы в процессе диффузии в жидкостях был достигнут определённый прогресс~\cite{FrenkelBook,HansenBook,GrootBook,MarchBook}, однако знания в данной области все еще довольно ограничены.
Существуют приблизительные соотношения, которые могут с разной степенью точности описать диффузию в различных системах.
Простейшей оценкой температурной зависимости коэффициента диффузии в жидкости является закон Аррениуса~\cite{10.1126/science.278.5336.257}.
Однако пренебрежение динамической вязкостью и особенностями реального потенциала взаимодействия делает данный закон непригодным для точного определения коэффициента диффузии в широком диапазоне температур.
Среди других соотношений диффузии в жидкостях, упомянем избыточное энтропийное масштабирование коэффициентов переноса~\cite{10.1103/physreva.15.2545, 10.1038/381137a0, 10.1063/1.5055064}, соотношение температуры замерзания и плотности~\cite{10.1103/physreve.62.7524, 10.1063/1.5022058, 10.1063/1.5044703, 10.1103/physreve.103.042122}, а также соотношение Стокс-Эйнштейна между коэффициентами вязкости диффузии и сдвига~\cite{10.1063/1.446338, 10.1002/BBPC.19900940313, 10.1103/physreve.95.052122, 10.1063/1.5080662, 10.1080/00268976.2019.1643045}.
Существуют методы, позволяющие достаточно точно прогнозировать коэфицент диффузии в конкретных системах, в том числе в широком интервале температур, вплоть до критической точки и в закритической области~\cite{10.1063/1.1607953, 10.1016/j.camwa.2019.11.012, 10.1063/1.441097}.
Получены обширные результаты численного моделирования~\cite{10.1063/1.1786579, 10.1016/j.fluid.2011.03.002}.
В настоящее время применяются методы машинного обучения~\cite{10.1063/5.0011512}.
Но остаются открытыми следующие важные вопросы:
\begin{enumerate}
\item Какое влияние имеет потенциал взаимодействия между частицами на температурную зависимость коэффициента диффузии;
\item Насколько важны корреляции между спектрами возбуждений и транспортными свойствами.
\end{enumerate}

\newpage

\textbf{Цель бакалаврской квалификационной работы} -- установить связь \\ дальнодействия притяжения потенциала взаимодействия и спектров возбуждений с транспортными свойствами жидкостей, а также выявить влияние дальнодействия притяжения на скорость нуклеации.

\textbf{Задачи бакалаврской квалификационной работы:}
\begin{enumerate}
\item Расчет фазовых диаграмм для 2D и 3D систем частиц, взаимодействующих посредством обобщенного потенциала Леннарда-Джонса с различными степенями притяжения. 
\item Адаптация метода кластеризации данных DBSCAN для изучения молекулярных систем и его сравнение с другими методами.
\item Расчет и анализ транспортных свойств и коллективных возбуждений на жидкостных бинодалях.
\item Применение нового метода распознавания фаз для изучения скорости нуклеации в переохлажденных системах Леннарда-Джонса с различным дальнодействием притяжения. 
\end{enumerate}


\textbf{Научной новизной обладают следующие результаты магистерской
  квалификационной работы:}
\begin{enumerate}
\item Установлено, что подвижность имеет линейную температурную зависимость в широком диапазоне на бинодали жидкость-газ.
\item При увеличении дальнодействия потенциала увеличивается отношение температур критической к тройной точке.
  Кроме того, при этом уменьшается наклон температурной зависимости подвижности.
\item Отклонение подвижности от линейной зависимости при высоких температурах коррелирует с переходом спектров возбуждений от осцилирующего к монотонному виду.
\end{enumerate}


\textbf{Апробация} основных результатов бакалаврской квалификационной работы проводилась на следующих конференциях:
\begin{enumerate}
  \item XVI ежегодный Молодежный Научный Форум «ТЕЛЕКОММУНИКАЦИИ И ИНФОРМАЦИОННЫЕ ТЕХНОЛОГИИ - РЕАЛИИ, ВОЗМОЖНОСТИ, ПЕРСПЕКТИВЫ», 5, 12 и 19 апреля 2025г.
  \item AINL: Artificial Intelligence and Natural Language Conference, HYBRID (Novosibirsk, Russia / ONLINE), 18-19 April 2025.
\end{enumerate}

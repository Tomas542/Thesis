\newpage
\begin{center}
  \textbf{\large АННОТАЦИЯ}
\end{center}


Объектом исследования данной работы стали большие языковые модели, а также модели распознавания речи и их взаимодействие.
С помощью данных моделей исследовались методы повышения точности транскрипций, перенос знаний о коррекции ошибок между разными моделями распознавания речи и наборами данных.
Для решения поставленной задачи методом пост-коррекции и рескоринга были обучены языковые модели.
Все эксперименты проводились с использованием языка Python и фреймворков Transformers, NeMO, KenLM Hydra.

В ходе работы были обучены и протестированы различные методы, выявлены их сильные и слабые стороны.
Посчитаны результаты на различных наборах данных.
В качестве основной метрики оценки использовалась Word Error Rate.
Предложен новый метод улучшения качества, способный работать с разными аудио моделями.
На основании полученных результатов, были сделаны вывод о переносе знаний об ошибках между разными моделями, пригодности языковых моделей для такого рода задач, влияние на скорость, точность и ресурсоёмкость задачи, а также предложены методы для дальнейшего улучшения метода.
Был разработан алгоритм обучения языковых моделей на основе ошибок систем распознавания речи и дальнейшего использования этих моделей для пост-коррекции.

Объём работы -- \textcolor{red}{TODO: количество страниц поправь}, количество источников -- \textcolor{red}{TODO: количество ссылок поправь}. Ключевые слова: распознавание речи, языковые модели, глубокое обучение, дообучение
\onehalfspacing

\newpage
\renewcommand{\contentsname}{\centerline{\large СОДЕРЖАНИЕ}}
\tableofcontents

\newpage
\begin{center}
  \textbf{\large ВВЕДЕНИЕ}
\end{center}
\addcontentsline{toc}{chapter}{ВВЕДЕНИЕ}


\textbf{Актуальность}

Распознание речи -- задача преобразование голосовой записи в текстовое представление.

Создание текстовых транскрипций для аудио  -- перспективная область, которая лежит в основе автоматизации многих процессов и играет ключевую роль в цифровой трансформации общества.
Сценариев использования этой технологии существует огромное количество в самых разных областях.

Распознание речи можно использовать в голосовых интерфейсах управления умными домами, роботами и мобильными устройствами, как это делают крупные технологические компании, такие как Сбер, Яндекс, ВК.
Или же для автоматического создания субтитров для любого аудио контента, такого как видеовстречи в компаниях, записи лекций, создание текстовых расшифровок звонков в колл-центрах.
Объёмы данных, включающих голос, растут каждый день.
Автоматизация различных процессов, завязанных на расшифровке, позволит эффективно масштабировать скорость обработки этих потоков данных.
Внедрение систем создания транскрипций для аудио ведёт к увеличению количества запросов, которые могут быть обработаны компаниями.
Проговорить задачи, отчёт или диагноз может быть намного проще и быстрее, чем напечатать его на клавиатуре или записать ручкой на бумаге.
Службы поддержки, построенные на таких системах способны работать без перерывов и предоставлять услуги в любое время дня и ночи.
Голос также может быть использован для оценки качества работы службы поддержки или же анализу доброжелательности клиентов и последующего использования этой информации для формирования рейтинга и возможного применения санкций по отношению к человеку.

Ещё одной важной областью является помощью людям с ограниченными возможностями.
Так для глухих людей в реальном времени может создаваться текстовая транскрипция.
В будущем при объединении этой технологии с умными очками позволит таким людям намного проще интегрироваться в общество и взаимодействовать с другими людьми.
Одним словом, улучшение точности распознавания речи положительно сказывается на многих отраслях.

Переход из аудио домена в лингвистический ведёт к расширению списка доступных задач, которые могут быть автоматизированы.
Так полученный текст далее можно использовать для создания краткой сводки по информации из записи, анализа настроений спикеров, автоматизация документооборота, автоматическое создание задач в системах мониторинга по итогам созвонов.
Или же аудио модели смогут сразу распознавать эмоции.
Ещё одним сценарием выступает создание тренажёра для изучения иностранных языков.
Модель распознавания речи может создавать транскрипцию речи пользователя, а потом с помощью языковой модели анализировать правильность произношения отдельных фраз или целых предложений.
Или же мы будем не просто расшифровывать звонок клиента, но вместе с тем на основе информации из звонка будем создавать автоматические обращения в нашу службу поддержки.
Такие сценарии использования позволят уменьшить количество ошибок, которые возникают из-за невнимательности сотрудников и прочих человеческих факторов.

Методы распознавания речи многие годы активно развиваются.
Благодаря развитию искусственного интеллекта и машинного обучения системы автоматического распознавания речи (ASR) стали точнее, быстрее и доступнее, что открывает новые возможности для бизнеса, государственного управления и повседневной жизни обычных граждан.
Пускай скорость изменений и нововведений в этой области пока что отстаёт от развития в других областях, связанных с текстовыми и визуальными данными, однако определённый прогресс есть.
Увеличение количества данных ведёт к росту возможностей методов глубокого обучения, основанных на текстовом или аудио доменах.
Это стало особенно востребовано последние годы с развитием больших языковых моделей, которым необходимы большие объёмы данных для дальнейшего обучения, расширения спектра задач и приобретения новых знаний.Поскольку качественные текстовые данные, которые находятся в интернете, конечны и по большей части уже находятся в использование, а LLM для дальнейшего для масштабирования нужно больше данных, одним из вариантов развития видят генерацию текстовых данных как раз из аудио.
Так OpenAI создали ASR модель Whisper для создания транскрипций голосовых данных и дальнейшего использования этих данных при обучении.

Также увеличение доступности вычислительных ресурсов и их мощности ведёт к созданию и развитию уже имеющихся больших и доступных моделей.
Современные решения можно запускать как локально, используя ресурсы потребительских видеокарт или же даже мощности мобильных устройств, так и в облаке с платой провайдерам за хостинг или расшифровку.
Всё тот же Whisper состоит из 1.5 млрд. параметров и способен распознавать речь на разных языках и при этом может быть запущен на графическом процессоре с 8 Гб видеопамяти.
Также помимо самой большой модели, есть и варианты 800 млн. параметров.
Другие компании также выкладывают модели, способные распознавать какой-то конкретный язык в общий доступ.
Так у Т-банка есть их VoiceKit, Сбер предоставляет возможность воспользоваться GigaAM, а Nvidia имеет в открытом доступе модель Fastconformer на 100 млн. параметров, которая может быть запущена на бюджетных видеокартах.

Последние годы многие специалисты в сфере искусственного интеллекта всё чаще исследуют проблему точности распознавания речи на разных языках.
Среди тем исследований наблюдается тенденция по увеличению числа решений, точность текстовых транскрипции которых всё чаще повышается не только за счёт новых архитектурных решений и добавления разнообразных данных в обучающую выборку, но также за счёт объединения систем распознавания речи с большими языковыми моделями.
Создание мультимодальным моделей вообще можно назвать трендом последних лет.
Во много основой этому является развитие языковых моделей, в частности декодер подходов на основе архитектуры General Purpose Transformer (GPT), объединение визуальной и аудио модальностей с текстовой через кодировщики и коннекторы.
Невозможно отрицать, что распознание речи будет играть одну из ключевых ролей у будущих омни агентов, которые будут имитировать человека.

Однако в настоящее время модели распознавания речи всё ещё не идеальны.
Они часто допускают ошибки в ходе работы, а эти ошибки в расшифровке могут вести к неправильной интерпретации эмоций или команды и, как следствие, нести экономические и репутационные издержки для разработчиков голосовых ассистентов или организаций, занимающихся голосовой аналитикой.
В областях вроде медицины такие ошибки могут стоить очень дорого.
Но остаются открытыми следующие важные вопросы:
\begin{enumerate}
\item Универсальность методов, возможность использовать его с разными моделями;
\item Насколько эффективно методы могут работать с различными темами и словами.
\end{enumerate}

\newpage

\textbf{Цель бакалаврской квалификационной работы} -- провести анализ предметной области и рассмотреть уже существующие модели и методов, выбрать из существующих вариантов наиболее хорошо работающие, дообучить языковую модель для коррекции транскрипций аудио, полученных из систем распознавания речи и провести сравнительный анализ работы.

\textbf{Задачи бакалаврской квалификационной работы:}
\begin{enumerate}
\item Посчитать прирост точности транскрипций аудио с использованием существующих методов. 
\item Дообучить языковую модель для адаптации под задачу пост-коррекции для русского языка.
\item Применить метод с различными конфигурациями для проверки универсальности его работы. 
\item Сравнить результаты с методами, заточенных сугубо под отдельные методы.
\end{enumerate}


\textbf{Научной новизной обладают следующие результаты бакалаврской
  квалификационной работы:}
\begin{enumerate}
\item Установлено, что языковые модели, обученные для одной системы расознания речи способны также эффективно исправлять транскрипции и для других ASR.
\item Тут добавить про PGD, как будут проведены соответствующие эксперименты.
\end{enumerate}


\textbf{Апробация} основных результатов бакалаврской квалификационной работы проводилась на следующих конференциях:
\begin{enumerate}
  \item XVI ежегодный Молодежный Научный Форум «ТЕЛЕКОММУНИКАЦИИ И ИНФОРМАЦИОННЫЕ ТЕХНОЛОГИИ - РЕАЛИИ, ВОЗМОЖНОСТИ, ПЕРСПЕКТИВЫ», 5, 12 и 19 апреля 2025г.
  \item AINL: Artificial Intelligence and Natural Language Conference, HYBRID (Novosibirsk, Russia / ONLINE), 18-19 April 2025.
\end{enumerate}

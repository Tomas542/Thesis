\newpage
\begin{center}
  \textbf{\large АННОТАЦИЯ}
\end{center}


Объектом исследования данной работы стали большие языковые модели, а также модели распознания речи и их взаимодействие.
С помощью данных моделей исследовались методы повышения точности транскрипций, перенос знаний о коррекции ошибок между разными моделями распознания речи и наборами данных.
Для решения поставленной задачи методом пост-коррекции и рескоринга были обучены языковые модели.
Все эксперименты проводились с использованием языка Python и фреймворков Transformers, NeMO, KenLM Hydra.

В ходе работы были обучены и протестированы различные методы, выявлены их сильные и слабые стороны.
Посчитаны результаты на различных наборах данных.
В качестве основной метрики оценки использовалась Word Error Rate.
Предложен новый метод улучшения качества, способный работать с разными аудио моделями.
На основании полученных результатов, были сделаны вывод о переносе знаний об ошибках между разными моделями, пригодности языковых моделей для такого рода задач, влияние на скорость, точность и ресурсоёмкость задачи, а также предложены методы для дальнейшего улучшения метода.
Был разработан алгоритм обучения языковых моделей на основе ошибок систем распознания речи и дальнейшего использования этих моделей для пост-коррекции.

\onehalfspacing
\setcounter{page}{2}

\newpage
\renewcommand{\contentsname}{\centerline{\large СОДЕРЖАНИЕ}}
\tableofcontents

\newpage
\begin{center}
  \textbf{\large ВВЕДЕНИЕ}
\end{center}
\addcontentsline{toc}{chapter}{ВВЕДЕНИЕ}


\textbf{Актуальность}

Распознание речи – задача преобразование голосовой записи в текстовое представление.
Транскрибация аудио в текст – перспективная область, которая лежит в основе автоматизации многих процессов.
Например, автоматическое создание субтитров для любого контента, содержащего аудио, создание текстовых расшифровок звонков в колл-центрах, управление умными домами с помощью голосового ассистента или раздача команд роботам.
Вместе с тем переход из аудио домена в лингвистический ведёт к расширению пула доступных задач.
Так полученный текст далее можно использовать для создания суммаризации, анализа настроений и т.д.

Ошибки в расшифровке могут вести к неправильной интерпретации эмоций или команды и, как следствие, нести экономические и репутационные издержки для создателей голосовых ассистентов или организаций, занимающихся голосовой аналитикой.
Дополнительно, объёмы данных, включащих голос, растут каждый день.
Автоматизация различных процессов, завязанных на расшифровке, позволит эффективно масштабировать скорость обработки этих потоков данных.
Также увеличение количества данных ведёт к росту возможностей методов глубокого обучения.

Методы распознания речи многие годы активно развиваются.
Пускай скорость этих самых изменений пока что отстаёт от развитися в областях, связанных с текстовыми и визуальными данными, однако определённый прогресс есть.
Особенно последние годы с развитием больших языковых моделей, которым необходимы большие объёмы данных для дальнейшего обучения.
Поскольку качественные текстовые данные, которые находятся в интернете, конечны, а LLM нужно больше данных для масштабирования, одним из вариантов развития видят генерацию текстовых данных как раз из аудио.
Так OpenAI создали модель Whisper для транскрибации голосовых данных и дальнейшего использования этих данных при обучение.

Также активное развитие доступности вычислительных ресурсов и их мощность ведёт к созданию больших и доступных моделей.
Всё тот же Whisper состоит из 1.5 млрд. параметров и способен распознавать речь на разных языках.
Другие эксперименты показывают возможности моделей сразу распознавать эмоции.
Одним словом улучшение точности распознания речи положительно сказывается на многих отраслях.

Послдение годы среди тендеций по повышению точности транскрибации аудио всё чаще подкрепляются не только новыми архитектурными решениями и добавлением разнообразных данных, но также объединением систем распознания речи с большими языковыми моделями.
Создание мультимодальным моделей вообще можно назвать трендом последних лет.
Во много основой этому является развитие языковых моделей, в частности декодер подходов на основе архитектуры GPT, объединение визуальной и аудио модальностей с текстовой через кодировщики и коннекторы.
Невозможно отрицать, что распознание речи будет играть одну из ключевых ролей у будущих омни агентов, которые будут иммитировать человека.

Однако в настоящее время модели распознания речи всё ещё не идеальны.
Они часто допускают ошибки в ходе работы.
Для решения этой проблемы существует множество различных методов, которые так или иначе улучшают точность транскрибации.
Но остаются открытыми следующие важные вопросы:
\begin{enumerate}
\item Универсальность методов, возможность использовать его с разными моделями;
\item Насколько эффективно методы могут работать с различными темами и словами.
\end{enumerate}

% За прошедшие годы в процессе диффузии в жидкостях был достигнут определённый прогресс~\cite{FrenkelBook,HansenBook,GrootBook,MarchBook}, однако знания в данной области все еще довольно ограничены.
% Существуют приблизительные соотношения, которые могут с разной степенью точности описать диффузию в различных системах.
% Простейшей оценкой температурной зависимости коэффициента диффузии в жидкости является закон Аррениуса~\cite{10.1126/science.278.5336.257}.
% Однако пренебрежение динамической вязкостью и особенностями реального потенциала взаимодействия делает данный закон непригодным для точного определения коэффициента диффузии в широком диапазоне температур.
% Среди других соотношений диффузии в жидкостях, упомянем избыточное энтропийное масштабирование коэффициентов переноса~\cite{10.1103/physreva.15.2545, 10.1038/381137a0, 10.1063/1.5055064}, соотношение температуры замерзания и плотности~\cite{10.1103/physreve.62.7524, 10.1063/1.5022058, 10.1063/1.5044703, 10.1103/physreve.103.042122}, а также соотношение Стокс-Эйнштейна между коэффициентами вязкости диффузии и сдвига~\cite{10.1063/1.446338, 10.1002/BBPC.19900940313, 10.1103/physreve.95.052122, 10.1063/1.5080662, 10.1080/00268976.2019.1643045}.
% Существуют методы, позволяющие достаточно точно прогнозировать коэфицент диффузии в конкретных системах, в том числе в широком интервале температур, вплоть до критической точки и в закритической области~\cite{10.1063/1.1607953, 10.1016/j.camwa.2019.11.012, 10.1063/1.441097}.
% Получены обширные результаты численного моделирования~\cite{10.1063/1.1786579, 10.1016/j.fluid.2011.03.002}.
% В настоящее время применяются методы машинного обучения~\cite{10.1063/5.0011512}.



\newpage

\textbf{Цель бакалаврской квалификационной работы} -- установить связь \\ дальнодействия притяжения потенциала взаимодействия и спектров возбуждений с транспортными свойствами жидкостей, а также выявить влияние дальнодействия притяжения на скорость нуклеации.

\textbf{Задачи бакалаврской квалификационной работы:}
\begin{enumerate}
\item Расчитать прирост точности транскрипций аудио с использованием существующих методов. 
\item Адаптация метода пост-коррекции для русского языка.
\item Применение метода с различными конфигурациями для проверки универсальности его работы. 
\end{enumerate}


\textbf{Научной новизной обладают следующие результаты бакалаврской
  квалификационной работы:}
\begin{enumerate}
\item Установлено, что языковые модели, обученные для одной системы расознания речи способны также эффективно исправлять транскрипции и для других моделей.
\item Тут добавить про PGD, как будут проведены соответствующие эксперименты.
\end{enumerate}


\textbf{Апробация} основных результатов бакалаврской квалификационной работы проводилась на следующих конференциях:
\begin{enumerate}
  \item XVI ежегодный Молодежный Научный Форум «ТЕЛЕКОММУНИКАЦИИ И ИНФОРМАЦИОННЫЕ ТЕХНОЛОГИИ - РЕАЛИИ, ВОЗМОЖНОСТИ, ПЕРСПЕКТИВЫ», 5, 12 и 19 апреля 2025г.
  \item AINL: Artificial Intelligence and Natural Language Conference, HYBRID (Novosibirsk, Russia / ONLINE), 18-19 April 2025.
\end{enumerate}
